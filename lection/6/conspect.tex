\documentclass[11pt,a4paper,oneside]{scrartcl}
\usepackage[utf8]{inputenc}
\usepackage[english,russian]{babel}
\usepackage[top=1cm,bottom=1cm,left=1cm,right=1cm]{geometry}
\usepackage{amsmath}
\usepackage{proof}
\usepackage{amsfonts}
\usepackage{amssymb}
\usepackage{amsthm}
\usepackage[inline]{enumitem}
\usepackage{hyperref}
\usepackage{mathtools}
\usepackage[T1]{fontenc}
\usepackage[normalem]{ulem}

\newtheorem{definition}{Определение}
\newtheorem{lemma}{Лемма}
\newtheorem*{theorem}{Теорема}
\newtheorem{consequence}{Следствие}
\newtheorem*{remark}{Замечание}
\newtheorem{ticket}{Билет}

\title{\href{https://www.youtube.com/live/T8Loz33oa0E?si=ALg7D2j652468pJr}{6-ая неделя}}

\date{9.10.2023}

\begin{document}
\pagestyle{empty}

\maketitle

% note: не соответствует порядку рассказа на лекции, но соответствует порядку скринов

\setcounter{ticket}{29}
\addtocounter{ticket}{-1}
\begin{ticket}[Перестановка пределов для рядов]
    Пусть $f_n : E \rightarrow \mathbb{C}$, $\sum_{n=1}^\infty f_n(x)$ сходится
    равномерно на $E$ и $\forall k \in \mathbb{N} \ \exists \lim_{x \rightarrow x_0} f_k(x)$.

    Тогда существуют оба и верно
    $\lim_{x \rightarrow x_0} \sum_{n=1}^\infty f_n(x)
    = \sum_{n=1}^\infty \lim_{x \rightarrow x_0} f_n(x)$
\end{ticket}

\section*{Билет 30. Следствия теоремы о перестановке пределов, связанные с непрерывностью
    (взято у Кости Баца, убрано доказательство)}

\begin{theorem}[Непрерывность в точке для последовательностей]
    $\sqsupset D\subseteq X$ -- м.п.,~ $\left\{ f_n \right\} , f:D \to \mathbb{C}$, ~
    $f_n\rightrightarrows f$ на  $D$.

    Если $\left\{ f_n \right\} $ непрерывны в точке $x_0$, то и $f$ непрерывна в $x_0$.
\end{theorem}

\begin{theorem}[Непрерывность в точке для рядов]
    Пусть $X$ -- м.п., $D \subset X, x_0 \in D, f_k: D \to \mathbb{R}
    ($или $\mathbb{C})$ и выполнены следующие условия:
    \begin{enumerate}
        \item ряд $\sum\limits_{k=1}^\infty$ равномерно сходится на $D$ к сумме $S$;
        \item все функции $f_k$ непрерывны в точке $x_0$.
    \end{enumerate}
    Тогда функция $S$ непрерывна в точке $x_0$.
\end{theorem}

\begin{theorem}[теорема Стокса-Зейделя]
    $D\subseteq X,\quad f_n, f: D \to \mathbb{C}\quad f_n\rightrightarrows f$ на $D$ при
    $n\to \infty $ и $f_n\in C(D) \implies f\in C(D)$, то есть
    равномерный предел последовательности непрерывных функций \textbf{непрерывен}.
\end{theorem}

\begin{theorem}[Аналог теоремы Стокса-Зейделя для рядов]
    $\sqsupset D\subseteq X$ -- м.п., $x_0\in  D^{'},   \left\{ f_n \right\}_{n=1}^{\infty },
    f:D \to \mathbb{C}$ и $\sum_{n=1}^{\infty } f_n(x)$ сходится равномерно на $D$.

    Если  $\forall n ~ f_n(x)$ непрерывна в точке $x_0$, то
    и $\sum_{n=1}^{\infty } f_n(x)$ непрерывно в $x_0$.
\end{theorem}

\section*{}

\setcounter{ticket}{32}
\addtocounter{ticket}{-1}
\begin{ticket}[Предельный переход под знаком интеграла для последовательностей]
    Если $f_n \in C[a, b]$, $f_n \rightrightarrows f$ на $[a, b]$,
    то $\int_a^b f(x)dx = \lim_{n \rightarrow \infty} \int_a^b f_n (x)dx$.
\end{ticket}

\addtocounter{ticket}{-1}
\begin{ticket}[Предельный переход под знаком интеграла для рядов]
    Если $f_n \in C[a, b]$, $\sum_{n=1}^\infty f_n$ сходится равномерно на $[a, b]$,
    то $\int_a^b \sum_{n=1}^\infty f_n(x) = \sum_{n=1}^\infty \int_a^b f_n(x)$,
    ряд в правой части сходится.
\end{ticket}

\setcounter{ticket}{33}
\addtocounter{ticket}{-1}
\begin{ticket}[Предельный переход под знаком производной для последовательностей]
    Пусть $f \in C^1([a, b] \rightarrow \mathbb{R})$,
    $\exists x^0 \in [a, b] : \{f_n(x^0)\}$ сходится при $n \rightarrow \infty$,
    $\{f_n'(x)\}$ равномерно сходится на $[a, b]$.

    Тогда $\lim_{n \rightarrow \infty} f_n(x)$ дифференцируема на $[a, b]$
    и $\forall x \in [a, b] \
    (\lim_{n \rightarrow \infty} f_n(x))' = \lim_{n \rightarrow \infty} f_n'(x)$.
\end{ticket}

\addtocounter{ticket}{-1}
\begin{ticket}[Предельный переход под знаком производной для рядов]
    Пусть $f \in C^1([a, b] \rightarrow \mathbb{R})$,
    $\exists x^0 \in [a, b] : \sum_{n=1}^\infty f_n(x^0)$ сходится,
    $\sum_{n=1}^\infty f_n'(x)$ равномерно сходится на $[a, b]$.

    Тогда $\sum_{n=1}^\infty f_n(x)$ дифференцируема на $[a, b]$
    и $\forall x \in [a, b] \ (\sum_{n=1}^\infty f_n(x))' = \sum_{n=1}^\infty f_n'(x)$.
\end{ticket}

\begin{ticket}[Теорема о круге сходимости степенного ряда]
    $a, \{c_n\}_{n=0}^\infty \in \mathbb{C}$, $\sum_{n=0}^\infty c_n (z-a)^n$ называется
    степенным рядом с коэффициентами $\{c_n\}$ и центром $a$.

    $B_r(a)$ называется кругом сходимости этого степенного ряда, если
    $\forall z \in B_r(a)$ ряд сходится и $\forall z \notin \overline{B}_r(a)$
    ряд расходится. $r$ называют радиусом сходимости.

    \textbf{Теорема Коши-Адамара.}
    $r = \frac{1}{\overline{\lim}_{n \rightarrow \infty} \sqrt[n]{|c_n|}}$.
    Тогда $r$ - радиус сходимости для степенного ряда.

    Точнее:
    \begin{enumerate}
        \item $\forall \text{компакта } K : K \subseteq B_r(a)$
            ряд сходится равномерно на $K$
        \item $\forall z \notin \overline{B}_r(a)$ ряд расходится в точке $z$
    \end{enumerate}

    При $r = \frac{1}{0}$ считаем $r = +\infty$, при $r = \frac{1}{+\infty}$,
    $r = 0$ (то есть круг сходимости содержит только центр).
\end{ticket}

\addtocounter{ticket}{-1}
\begin{ticket}[Формулы для радиуса сходимости]
    Кажется, одна из формул в целом и является предыдущей теоремой.
    Вот другая формула: $\lim_{n \rightarrow \infty} \frac{|c_n|}{|c_{n+1}|}$
    (в случае существования).
\end{ticket}

\setcounter{ticket}{19}
\addtocounter{ticket}{-1}
\begin{ticket}[Параметризации поверхностей, гладкие поверхности уровня, гладкие обобщенные графики]

    $M \subseteq \mathbb{R}^{n+m}$, $a \in M$, $M$ допускает параметризацию класса $C^r$
    размерности $n$ в окрестности $a$, если $\exists \text{окрестность } U_a,
    \text{гомеоморфизм } \Phi \in C^r (\mathbb{R}^n \supseteq O \rightarrow U_a \cap M)$,
    $\Phi$ регулярное.

    $M \subseteq \mathbb{R}^{n+m}$, $a \in M$, $M$ есть множество уровня класса $C^r$
    размерности $n$ в окрестности $a$, если $\exists \text{окрестность } U_a,
    F \in C^r(U_a \rightarrow \mathbb{R}^m)$, $F$ регулярно, $M \cap U_a = \{x \in U_a : F(x) = 0\}$.

    $M \subseteq \mathbb{R}^{n+m}$, $a \in M$, $M$ есть обобщенный $r$-гладкий график
    размерности $n$ в окрестности $a$, если \\ $\exists \text{окрестность } U_a,
    f \in C^r (\mathbb{R}^n \supseteq O \rightarrow \mathbb{R}^m) : U_a \cap M = \Gamma_f$
    с точностью до перестановки координат.
\end{ticket}

\addtocounter{ticket}{-1}
\begin{ticket}[Теорема о способах задания k-мерной поверхности]
    Пусть $m, n \in \mathbb{N}$, $M \subseteq \mathbb{R}^{n+m}$, $a \in M$.

    Тогда следующие утверждения равносильны:
    \begin{enumerate}
        \item В окрестности $a$ $M$ - $n$-мерный $C^r$-гладкий обобщенный график
        \item В окрестности $a$ $M$ - $n$-мерное $C^r$-гладкое множество уровня
        \item В окрестности $a$ $M$ допускает $n$-мерную $C^r$-гладкую параметризацию
    \end{enumerate}
\end{ticket}

\end{document}
