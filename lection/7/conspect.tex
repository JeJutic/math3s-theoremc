\documentclass[11pt,a4paper,oneside]{scrartcl}
\usepackage[utf8]{inputenc}
\usepackage[english,russian]{babel}
\usepackage[top=1cm,bottom=1cm,left=1cm,right=1cm]{geometry}
\usepackage{amsmath}
\usepackage{proof}
\usepackage{amsfonts}
\usepackage{amssymb}
\usepackage{amsthm}
\usepackage[inline]{enumitem}
\usepackage{hyperref}
\usepackage{mathtools}
\usepackage[T1]{fontenc}
\usepackage[normalem]{ulem}

\newcommand{\Z}{\mathbb{Z}}
\newcommand{\R}{\mathbb{R}}
\newcommand{\Cx}{\mathbb{C}}
\newcommand{\Lagr}{\mathcal{L}}
\newcommand{\M}{\mathcal{M}}

\newtheorem{definition}{Определение}
\newtheorem{lemma}{Лемма}
\newtheorem*{theorem}{Теорема}
\newtheorem{consequence}{Следствие}
\newtheorem*{remark}{Замечание}
\newtheorem{ticket}{Билет}

\title{\href{https://www.youtube.com/live/pxLv7b4ne-4?si=FdHOL9RRxvanex11}{7-ая неделя}}

\date{16.10.2023}

\begin{document}
\pagestyle{empty}

\maketitle

% note: не соответствует порядку рассказа на лекции, но соответствует порядку скринов

\setcounter{ticket}{34}
\addtocounter{ticket}{-1}
\begin{ticket}[(с леммой о верхнем пределе произведения)]
    $x_n, y_n \in \mathbb{R}$, $x_n \rightarrow x$, $x > 0$.

    Тогда $\overline{\lim}_{n \rightarrow \infty} x_n y_n
    = x \overline{\lim}_{n \rightarrow \infty} y_n$.
\end{ticket}

\begin{consequence}[Сумма степенного ряда непрерывна в круге сходимости]
    skipped
\end{consequence}

\begin{ticket}[Теорема Абеля]
    $R$ - радиус сходимости $\sum_{n=0}^\infty c_n x^n$, $R > 0$.

    Тогда
    \begin{enumerate}
        \item Если ряд сходится в точке $R$, то он сходится равномерно на $[0, R]$
        \item Если ряд сходится в точке $-R$, то он сходится равномерно на $[-R, 0]$
    \end{enumerate}
\end{ticket}

\addtocounter{ticket}{-1}
\begin{ticket}[Интегрирование степенных рядов]
    $[\alpha, \beta] \subset (a-r, a+r)$, $r$ - радиус сходимости
    степенного ряда $\sum_{n=0}^\infty c_n (z-a)^n$.

    Тогда $\int_\alpha^\beta \sum_{n=0}^\infty c_n (x-a)^n dx
    = \sum_{n=0}^\infty c_n \int_\alpha^\beta (x-a)^n dx$,
    то есть ряд допускает почленное интегрирование.
\end{ticket}

\addtocounter{ticket}{-1}
\begin{ticket}[Дифференцирование степенных рядов]
    Степенной ряд $\sum_{n=0}^\infty c_n (z-a)^n \in C^\infty (B_r (a))$,
    где $r$ - радиус сходимости.

    Этот ряд допускает m-кратное дифференцирование почленно $\forall m \in \Z_+$
    и $(\sum_{n=0}^\infty c_n (z-a)^n)^{(m)}
    = \sum_{n=m}^\infty n (n-1) \dots(n-m+1) c_n (z-a)^{n-m}$, $z \in B_r(a)$
\end{ticket}

\setcounter{consequence}{0}
\begin{consequence}
    Пусть $[\alpha, \beta] \subset (a - r, a + r)$, где
    $r = \frac{1}{\overline{\lim}_{n \rightarrow \infty} \sqrt[n]{|c_n|}}$

    Тогда $\int_\alpha^\beta \sum_{n=0}^\infty c_n (x-a)^n dx
    = \sum_{n=0}^\infty c_n \int_\alpha^\beta (x-a)^n dx$,
    то есть ряд допускает почленное дифференцирование на $[\alpha, \beta]$.

    Если ряд сходится в точке $a + r$ (или $a - r$), то утверждение верно
    и для $[\alpha, \beta] \subseteq (a - r, a + r]$
    (или $[\alpha, \beta] \subseteq [a - r, a + r)$)
\end{consequence}

\begin{definition}[Комплексная дифференцируемость]
    $f: \Cx \supseteq O \rightarrow \Cx$, $a \in O$

    $f'(a) = \lim_{z \rightarrow a} \frac{f(z) - f(a)}{z - a}$
    - производная $f$ в точке $a$ (если предел существует).
\end{definition}

\begin{theorem}
    Степенной ряд $\sum_{n=0}^\infty c_n (z-a)^n \in C^\infty (B_r(a))$,
    где $r$ - радиус сходимости.

    Этот ряд допускает $m$-кратное дифференцирование почленно $\forall m \in \Z_+$
    и $(\sum_{n=0}^\infty c_n (z-a)^n)^{(m)} =
    \sum_{n=m}^\infty n (n-1) \dots (n-m+1) c_n (z-a)^{n-m}$, $z \in B_r(a)$
\end{theorem}

\setcounter{ticket}{21}
\addtocounter{ticket}{-1}
\begin{ticket}[Необходимое условие условного экстремума (геометрическая формулировка)]
    $m, N \in \mathbb{N}$, $m < N$, $\mathbb{R}^n \supseteq O$ открытое,
    $F_1, \dots, F_m, f \in C^1(O)$ и $F = (F_1, \dots, F_m)$, $F$ регулярно в $O$;
    $a \in O$, $a$ - точка условного экстремума $f$ при условие $F(x) = 0$.

    Тогда $\nabla_a f$ есть линейная комбинация $\nabla_a F_1, \dots, \nabla_a F_m$,
    то есть $\exists \lambda_1, \dots, \lambda_m:
        \nabla_a f = \sum_{k=1}^m \lambda_k \cdot \nabla_a F_k$.
\end{ticket}

\addtocounter{ticket}{-1}
\begin{ticket}[Необходимое условие условного экстремума (формулировка,
        использующая функцию Лагранжа)]
    $\Lagr(x_1, \dots, x_n, \lambda_1, \dots, \lambda_m)
        = f(x_1, \dots, x_n) - \sum_{k=1}^m \lambda_k F_k(x_1, \dots, x_n)$
    - функция Лагранжа, отвечающая функции $f$ и системе связи $F_i(x) = 0$.

    Пусть выполнено условие формулировки выше.

    Тогда $\exists \lambda \in \R^m : d_{(a, \lambda)} \Lagr = 0$.
\end{ticket}

\setcounter{ticket}{20}
\addtocounter{ticket}{-1}
\begin{ticket}[Линейное касательное пространство к k-мерной поверхности — определение \sout{и свойства}]
    $\M \subseteq \R^n$, $p \in \M$, $\tau \in \R^N$, $\tau$ называется
    касательным вектором к $\M$ в точке $p$ ($p \in T_p \M$),
    если $\exists$ гладкое отображение $\gamma: (a, b) \rightarrow \M$
    и $\exists c \in (a, b): \gamma(c) = p, \gamma'(c) = \tau$.
\end{ticket}

\addtocounter{ticket}{-1}
\begin{ticket}[Канонические базисы линейного касательного пространства]
    $\M$ допускает (в окрестности точки $p$) гладкую параметризацию
    $\Phi: \R^n \supseteq U \rightarrow \R^N$, $\Phi(U) = \M$;
    $a \in U$, $\Phi(a) = p$.

    Тогда $\frac{\partial \Phi}{\partial x_1}(a), \dots,
        \frac{\partial \Phi}{\partial x_n}(a)$
    называются каноническими касательными векторами.
    Если $\Phi$ регулярно в точке $a$, они линейно независимы.

%    $\gamme_j(t) \coloneqq a + t e_j$, где $e_j$ - j-ый базисный вектор $\R^n$
%    и $t \in (-\delta, \delta)$. Заметим, что отображения гладкие.
%
%    $\M \subseteq \R^n$, $p \in \M$, $\tau \in \R^N$, $\tau$ называется
%    касательным вектором к $\M$ в точке $p$ ($p \in T_p \M$),
%    если $\exists$ гладкое отображение $\gamma: (a, b) \rightarrow \M$
%    и $\exists c \in (a, b): \gamma(c) = p, \gamma'(c) = \tau$.
\end{ticket}

\addtocounter{ticket}{-1}
\begin{ticket}[и его ортогонального дополнения]
    По теореме о способах задания гладких многообразий % не вошла в скрин
    $\exists F: \R^n \supseteq O \text{ (открытое) } \rightarrow \R^m$, $m + n = N$,
    $\M \cap O = \{x: F(x) = 0\}$.

    Тогда $\forall \tau \in T_p \M \ \forall j=1, \dots, m \ \tau \perp \nabla_p F_j$
    и $\{\nabla_p F_j\}_{j=1}^m$ является каноническим базисом ортогонального
    дополнения линейного касательного пространства:
    $T_p \M = (span(\nabla_p F_1, \dots, \nabla_p F_m))^\perp$.
\end{ticket}





\end{document}
