\documentclass[11pt,a4paper,oneside]{scrartcl}
\usepackage[utf8]{inputenc}
\usepackage[english,russian]{babel}
\usepackage[top=1cm,bottom=1cm,left=1cm,right=1cm]{geometry}
\usepackage{amsmath}
\usepackage{proof}
\usepackage{amsfonts}
\usepackage{amssymb}
\usepackage{amsthm}
\usepackage[inline]{enumitem}
\usepackage{hyperref}
\usepackage{mathtools}
\usepackage[T1]{fontenc}

\newtheorem{definition}{Определение}
\newtheorem{theorem}{Теорема}
\newtheorem{consequence}{Следствие}
\newtheorem*{remark}{Замечание}

\title{\href{https://www.youtube.com/live/0p9G5WBcDqs?si=1_87BezxIHYyR8rZ}{1-ая неделя}}

\date{4.09.2023}

\begin{document}
\pagestyle{empty}

\maketitle

\begin{theorem}[Необходимое условие дифференцируемости]
    Если $f: \mathbb{R}^n \supseteq O \rightarrow \mathbb{R}^m$
    дифференцируема в точке $a$,
    то $\forall u \in \mathbb{R}^n \, \exists \frac{\partial f}{\partial u}(a)$
    (далее показано, что это эквивалентно для частных производных только по $x_i$).
\end{theorem}

\begin{theorem}[Дифференциал композиции]    % 20m
    Пусть $g: X \rightarrow Y$, $f: Y \rightarrow Z$.
    Тогда если $g$ дифференцируема в точке $a$
    и f дифференцируема в точке $g(a)$,
    то $f \circ g$ дифференцируема в точке $a$
    и $d_a (f \circ g) = d_{g(a)} f \cdot d_a g$.

    Или, если рассматривать матрицу Якоби, $(f \circ g)'(a) = f'(g(a)) \cdot g'(a)$
\end{theorem}

\begin{theorem}[Дифференцирование результата арифметических действий]    % 46m
    Пусть $O \subseteq \mathbb{R}^n$, $a \in O$;
    $f, g: O \rightarrow \mathbb{R}^m$, $\lambda: O \rightarrow R$;
    $f, g, \lambda$ дифференцируемы в точке $a$;
    $A, B \in \mathbb{R}$.

    Тогда \begin{enumerate}
              \item $Af + Bg$ дифференцируемо в точке $a$ и $d_a (Af + Bg) = Ad_a f + Bd_a g$
              \item $\lambda f$ дифференцируемо в точке $a$
                и $d_a (\lambda f) = f(a) \cdot d_a \lambda + \lambda(a) \cdot d_a f$ \\
                Или на языке матриц:
                    $(\lambda f)' = f(a) \cdot \lambda'(a) + \lambda(a) \cdot f'(a) $
              \item $\langle f, g \rangle$ дифференцируемо в точке $a$ и
                $d_a \langle f, g \rangle = (g(a))^T d_a f + (f(a))^T d_a g $ \\
                $(\langle f, g \rangle)' = (g(a))^T \cdot f'(a) + (f(a))^T \cdot g'(a)$
              \item Если $m = 1$ и $g(a) \neq 0$, то $f / g$ дифференцируемо в точке $a$
                и $d_a (f / g) = \frac{g(a)d_a f - f(a)d_a g}{g^2(a)}$
    \end{enumerate}
    если $g$ дифференцируема в точке $a$
    и f дифференцируема в точке $g(a)$,
    то $f \circ g$ дифференцируема в точке $a$
    и $d_a (f \circ g) = d_{g(a)} f \cdot d_a g$.

    Или, если рассматривать матрицу Якоби, $(f \circ g)'(a) = f'(g(a)) \cdot g'(a)$
\end{theorem}

\begin{theorem}[Теорема Лагранжа для отображений]    % 1:15
    Пусть $f: \mathbb{R}^n \supseteq O \text{(открытое)} \rightarrow \mathbb{R}^m$,
    $f$ дифференцируемо в $O$; $a, b \in O$, $\forall t \in (0, 1) \ a + t(b - a) \in O$.

    Тогда $\exists \theta \in (0, 1): ||f(b) - f(a)|| \leq ||f'(a + \theta(b - a))|| \cdot ||b-a||$
\end{theorem}

\setcounter{consequence}{0}
\begin{consequence}
    Если $\forall \theta \in (0, 1) \ ||f'(a + \theta(b - a))|| \leq M \in \mathbb{R}$,
    то $||f(b) - f(a)|| \leq M ||(b - a)||$
\end{consequence}
\begin{consequence}
    Если $m = 1$
    и $\forall u \in O \ \forall i = 1..n \ ||\frac{\partial f}{\partial x_i}(u)|| \leq M$,
    то $||f(b) - f(a)|| \leq M \sqrt{n} ||(b - a)||$
\end{consequence}

\begin{theorem}[Достаточное условие дифференцируемости]    % 1:36
    Пусть $f: \mathbb{R}^n \supseteq O \text{(открытое)} \rightarrow \mathbb{R}^m$, $a \in O$;
    $\frac{\partial f}{\partial x_i} \ \forall i \in 1..n$
    \begin{enumerate*} [label=\itshape\arabic*\upshape)]
        \item определен в некоторой окрестности точки $a$
        \item непрерывен в точкe $a$
    \end{enumerate*}

    Тогда $f$ дифференцируема в точке $a$.
\end{theorem}

\begin{remark}
    $f$ дифференцируема в точке $a$
    $\Leftrightarrow f(a + h) - f(a) - f'(a) \cdot h = o(h)$ при $h \rightarrow 0$
\end{remark}

\begin{definition}
    Пусть $f: \mathbb{R}^n \supseteq O \text{(открытое)} \rightarrow \mathbb{R}$,
    $g(u) = \frac{\partial f}{\partial x_i}(u)$ для некоторого $i$ определена в точке $a$
    и $\exists \frac{\partial g}{\partial x_j}(a)$ для некоторого $j$.

    Тогда $f_{x_i x_j} \coloneqq \frac{\partial^2 f}{\partial x_j \partial x_i}(a)
        \coloneqq \frac{\partial g}{\partial x_i}(a)$
\end{definition}

\begin{definition}
    $\frac{\partial^2 f}{\partial x_i^2} \coloneqq \frac{\partial^2 f}{\partial x_j \partial x_i}$
        - чистая частная производная.
\end{definition}

\begin{definition}
    $f_{x_i x_j}$, где $i \neq j$, - смешанная производная.
\end{definition}

\begin{theorem}
    Пусть $f: \mathbb{R}^n \supseteq O \text{(открытое)} \rightarrow \mathbb{R}$, $i \neq j$;
    $\frac{\partial^2 f}{\partial x_j \partial x_i}$
    и $\frac{\partial^2 f}{\partial x_i \partial x_j}$
    определены и непрерывны в окрестности точка $a$.

    Тогда $\frac{\partial^2 f}{\partial x_j \partial x_i}(a)
        = \frac{\partial^2 f}{\partial x_i \partial x_j}(a)$
\end{theorem}

\begin{definition}
    Если $f: \mathbb{R}^n \supseteq O \rightarrow \mathbb{R}$, $h \in \mathbb{R}^n$,
    то $d_a^2 f(h) \coloneqq d(d_a f(h))(h)$
\end{definition}

\end{document}
