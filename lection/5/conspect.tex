\documentclass[11pt,a4paper,oneside]{scrartcl}
\usepackage[utf8]{inputenc}
\usepackage[english,russian]{babel}
\usepackage[top=1cm,bottom=1cm,left=1cm,right=1cm]{geometry}
\usepackage{amsmath}
\usepackage{proof}
\usepackage{amsfonts}
\usepackage{amssymb}
\usepackage{amsthm}
\usepackage[inline]{enumitem}
\usepackage{hyperref}
\usepackage{mathtools}
\usepackage[T1]{fontenc}
\usepackage[normalem]{ulem}

\newtheorem{definition}{Определение}
\newtheorem{lemma}{Лемма}
\newtheorem*{theorem}{Теорема}
\newtheorem{consequence}{Следствие}
\newtheorem*{remark}{Замечание}
\newtheorem{ticket}{Билет}

\title{\href{https://youtu.be/KuG517F82VE?si=f8m6ZjJHx0_MvQsf}{5-ая неделя}}

\date{2.10.2023}

\begin{document}
\pagestyle{empty}

\maketitle

\setcounter{ticket}{15}
\addtocounter{ticket}{-1}
\begin{ticket}[Теорема о дифференцируемости обратного отображения]
    Пусть $E \subseteq \mathbb{R}^n$, $a \in IntE$, $\Phi: E \rightarrow \mathbb{R}^n$,
    $\Phi(a) = b \in Int \Phi(E)$, $\Phi$ дифференцируема в $a$,
    $\Phi'(a)$ обратима ($\det \Phi'(a) \neq 0$).

    Тогда $\Phi^{-1}$ дифференцируема в $b$ и $(\Phi^{-1})'(b) = (\Phi'(a))^{-1}$
\end{ticket}

\begin{ticket}[Теорема о гладкости обратного отображения (достаточное условие диффеоморфности)]
    $O, \tilde{O}$ открытые, $\Phi: O \rightarrow \tilde{O}$ - диффеоморфизм на $C^r$ $\xRightarrow{def} \Phi$ обратима
    и $\Phi \in C^r(O \rightarrow \tilde{O})$, $\Phi^{-1} \in C^r(\tilde{O} \rightarrow O)$.

    Если $O$ - открытое, $O \subseteq \mathbb{R}^n$, $\Phi \in C^r(O \rightarrow \mathbb{R}^n)$,
    $\Phi$ обратимо (как отображение на свой образ) и $\det \Phi'(x) \neq 0$ всюду в $O$.

    Тогда $\Phi^{-1} \in C^r(\Phi(O) \rightarrow O)$
    ($\forall x \in O (\Phi^{-1})(\Phi(x)) = (\Phi'(x))^{-1}$)
\end{ticket}

\begin{ticket}[Теорема о локальной обратимости регулярного отображения]
    $\Phi: \mathbb{R}^n \supseteq O \rightarrow \mathbb{R}^n$, $O$ открытое;
    $\Phi$ регулярное $\xRightarrow{def} \Phi \in C^1(O \rightarrow \mathbb{R}^n)$,
    $rank \Phi'(x)$ максимальной в каждой точке $O$.

    Пусть $\mathbb{R}^n \supseteq O$ открытое, $\Phi \in C^r(O \rightarrow \mathbb{R}^n)$,
    $\Phi$ регулярно в $O$.

    Тогда $\forall a \in O \ \exists \text{окрестность } U_a: \Phi |_{U_a}$
    - диффеоморфизм класса $C^r$, в частности обратимо.
\end{ticket}

\begin{ticket}[Теорема о неявном отображении]
    Пусть $m, n, r \in \mathbb{N}$, $\mathbb{R}^{n+m} \supseteq O$ открытое,
    $x \in \mathbb{R}^n$, $y \in \mathbb{R}^m$, $x^0 \in \mathbb{R}^n$, $y^0 \in \mathbb{R}^m$,
    $F \in C^r(O \rightarrow \mathbb{R}^m)$
    и $F'$ обратима.

    Тогда $\exists \text{окрестности } U_{x^0}, U_{y^0}$ и $f: U_{x^0} \rightarrow U_{y^0}$
    такие, что:
    \begin{enumerate}
        \item $F(x, y) = 0 \Leftrightarrow y = f(x)$ в $U_{x^0} \times U_{y^0}$
        \item $f \in C^r(U_{x^0} \rightarrow U_{y^0})$
        \item $f'(x) = -(F_y'(x, f(x)))^{-1} \cdot F_x'(x, f(x))$
    \end{enumerate}
\end{ticket}

\setcounter{ticket}{27}
\addtocounter{ticket}{-1}
\begin{ticket}[(с леммой) (версия по лекции)]
    $\sum_{k=1}^\infty f_k(x)$ равномерно сходится на $E$, $\varphi(x)$ ограничен
    на $E \Rightarrow \sum_{k=1}^\infty \varphi(x) f_k(x)$
    равномерно сходится на $E$
\end{ticket}

\begin{ticket}[\sout{Примеры исследования рядов на равномерную сходимость}]
\end{ticket}

\begin{theorem}[Признак Лейбница равномерной сходимости]
    Skipped
\end{theorem}

\begin{theorem}[Признак равномерной сходимости для монотонных последовательностей]
    Skipped
\end{theorem}

\begin{ticket}[Перестановка пределов для последовательностей]
    Пусть $E \in \mathbb{R}^n$, $x_0 \in E$, $f_n : E \rightarrow \mathbb{C}$,
    ${f_n (x)}$ равномерно сходится на $E$, $\forall k \in N \
    \exists \lim_{x \rightarrow x_0} f_k(x) \in \mathbb{R}$.

    Тогда $\lim_{x \rightarrow x_0} \lim_{k \rightarrow \infty} f_k(x) =
    \lim_{k \rightarrow \infty} \lim_{x \rightarrow x_0} f_k(x)$,
    оба предела существуют в $\mathbb{R}$.
\end{ticket}

\setcounter{consequence}{0}
\begin{consequence}
    Skipped
\end{consequence}

\end{document}
